\documentclass[twocolumn, a4paper]{article} % 核心:双栏模式

% --- 导入必要的包 ---
\usepackage{graphicx}  % 用于插入图片
\usepackage{booktabs}  % 用于美化表格
\usepackage{amsmath}   % 用于数学公式
\usepackage[margin=2cm]{geometry} % 设置页边距

% --- 元数据区域 (测试 Header 提取) ---
\title{Evaluation of PDF Parsers: A Minimal Test Case}
\author{San Zhang$^{1}$, Si Li$^{2}$ \\ 
        \small $^{1}$Department of Computer Science, Test University \\
        \small $^{2}$Artificial Intelligence Lab, Demo Institute}
\date{February 2026}

\begin{document}

\maketitle

% --- 摘要 (测试 Abstract 提取) ---
\begin{abstract}
This document is a minimal example designed to test the capabilities of PDF parsing tools like GROBID and MinerU. It contains mixed layouts including figures, tables, and lists within a double-column format.
\end{abstract}

% --- 正文区 ---

\section{Introduction}
This section tests the parser's ability to handle standard text blocks. The document layout is set to \textbf{double-column}, which is common in academic conferences (e.g., CVPR, ACL). 
Parsers must correctly identify the reading order (left column first, then right column).

\section{Formatting Tests}

\subsection{Vertical Lists}
Here is a vertical list to test structure recognition:
\begin{itemize}
    \item \textbf{Item 1:} First component of the list.
    \item \textbf{Item 2:} Second component with value $x$.
    \item \textbf{Item 3:} Final component of the test.
\end{itemize}

\subsection{Mathematical Formulas}
Parsers should identify the following display equation independently from the text:
\begin{equation}
    E = mc^2 + \sum_{i=0}^{n} \alpha_i x_i
    \label{eq:test}
\end{equation}
Where $E$ represents energy and $m$ represents mass.

% --- 图片测试 (使用内置示例图,无需上传) ---
\begin{figure}[h]
    \centering
    % example-image 是 LaTeX 内置的占位图
    \includegraphics[width=0.5\linewidth]{example-image-a}\includegraphics[width=0.5\linewidth]{example-image-b}
    \caption{A test figure. Parsers should extract this caption and associate it with the image above.}
    \label{fig:test}
\end{figure}

\section{Table Analysis}
This section forces the parser to handle structured data grids. 

\begin{table}[h]
    \centering
    \caption{Comparison of different models}
    \begin{tabular}{lcc}
        \toprule
        \textbf{Model} & \textbf{Accuracy} & \textbf{Speed (ms)} \\
        \midrule
        Base Model    & 85.4\%           & 12 \\
        Large Model   & 92.1\%           & 45 \\
        \textbf{Our Method} & \textbf{94.5\%} & \textbf{15} \\
        \bottomrule
    \end{tabular}
    \label{tab:results}
\end{table}

\section{Conclusion}
This concludes the minimal test. The parser should recognize this as the final section before references.

% --- 图片测试 ---
\begin{figure}[h]
    \centering
    % 注意:花括号里填的是你上传后的文件名,不要填 URL!
    \includegraphics[width=1.0\linewidth]{test_image.png} 
    \caption{Comparison of RAG architectures (Extracted from image).}
    \label{fig:rag_arch}
\end{figure}

% --- 参考文献 (测试 Citation 提取) ---
\begin{thebibliography}{9}
    \bibitem{vaswani2017}
    A. Vaswani et al., "Attention is all you need," in \textit{NIPS}, 2017.
    
    \bibitem{resnet2016}
    K. He, X. Zhang, S. Ren, and J. Sun, "Deep residual learning for image recognition," in \textit{CVPR}, 2016.
\end{thebibliography}

\end{document}