\documentclass[a4paper, 12pt]{article} % 单栏模式

% --- 导入必要的包 ---
\usepackage{graphicx}   % 用于插入图片
\usepackage{booktabs}   % 用于美化表格
\usepackage{amsmath}    % 用于数学公式
\usepackage{rotating}   % 用于竖排文字
\usepackage{float}      % 【关键修改1】引入强力定位包,防止图片乱跑
\usepackage[margin=2.5cm]{geometry} % 设置页边距

% --- 元数据区域 ---
\title{Comprehensive Parser Test: Single Column Layout}
\author{Test Engineer}
\date{February 2026}

\begin{document}

\maketitle

% --- 1. 摘要 ---
\begin{abstract}
This document serves as a minimal yet comprehensive test case for PDF parsers. It is formatted in a single-column layout and includes lists, tables, images, and non-standard text orientations.
\end{abstract}

% --- 2. 列表测试 ---
\section{List Structures}
Parsers should identify the following items as a structured list.
\begin{itemize}
    \item \textbf{Level 1 Item:} Standard bullet point.
    \begin{itemize}
        \item \textbf{Level 2 Item:} Nested bullet point.
    \end{itemize}
    \item \textbf{Level 1 Item:} Back to main level.
\end{itemize}

% --- 3. 竖排文字测试 ---
\section{Vertical Text Layout}
\noindent
\begin{minipage}{0.8\textwidth}
    \textbf{Normal Horizontal Text:} \\
    This is a standard paragraph block. The parser should read this continuously from left to right.
\end{minipage}%
\hfill
\begin{minipage}{0.15\textwidth}
    \rotatebox{90}{\textbf{>>> Vertical Text <<<}}
\end{minipage}

% --- 4. 表格测试 ---
\section{Table Data Extraction}
\begin{table}[h]
    \centering
    \caption{Performance Benchmarks}
    \begin{tabular}{l c c}
        \toprule
        \textbf{Model Name} & \textbf{Precision} & \textbf{Recall} \\
        \midrule
        Model A  & 88.5\% & 81.2\% \\
        Model B  & 92.4\% & 89.6\% \\
        \bottomrule
    \end{tabular}
    \label{tab:perf}
\end{table}

% --- 5. 图片测试 (关键修改区域) ---
\section{Figure Handling}
Below are two figures. They should appear strictly in this order.

% 第一张图:内置示例图
% 【关键修改2】把 [h] 改成了 [H],强制固定位置
\begin{figure}[H] 
    \centering
    \includegraphics[width=0.6\textwidth]{example-image}
    \caption{Standard test image (Figure 1). Extracted text should include this caption.}
    \label{fig:main}
\end{figure}

% 第二张图:你的 RAG 架构图
% 【关键修改3】同样使用 [H],确保它紧接着上一张图,不会跑到参考文献后面
\begin{figure}[H] 
    \centering
    % 请确保文件名正确,且图片已上传
    \includegraphics[width=1.0\textwidth]{test_image.png} 
    \caption{Comparison of RAG architectures (Figure 2).}
    \label{fig:rag_arch}
\end{figure}

% --- 6. 数学公式 ---
\section{Equation}
\begin{equation}
    f(x) = \int_{0}^{\infty} e^{-x^2} dx
\end{equation}

% --- 7. 参考文献 ---
\begin{thebibliography}{9}
    \bibitem{test2026}
    J. Doe, "Parsing Vertical Text in PDFs," \textit{Journal of Testing}, 2026.
\end{thebibliography}

\end{document}